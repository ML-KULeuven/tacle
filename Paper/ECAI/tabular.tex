\documentclass{ecai}
\usepackage{times}
\usepackage{graphicx}
\usepackage{latexsym}
\usepackage{xspace}
\usepackage[numbers]{natbib}
\usepackage{notoccite}

\usepackage[dvipsnames]{xcolor}
\newcommand{\sergey}[1]{\textcolor{magenta}{{\sc Sergey} #1}\xspace}

%%\ecaisubmission   % inserts page numbers. Use only for submission of paper.
                  % Do NOT use for camera-ready version of paper.

\begin{document}

\title{Tabular Constraint Learning}

\author{Name1 Surname1 \and Name2 Surname2 \and Name3 Surname3 \institute{KU Leuven, Belgium, email: firstname.lastname@kuleuven} }

\maketitle

\begin{abstract}
  abstract
  \end{abstract}
\section{Introduction}
\sergey{bullet points for luc to start introduction}\\
\textbf{Key question}:\\
Can we discover or reconstruct structural relations in flat tabular spreadsheet data? [in a general way that allows declarative specification of constraints to discover]\


\textbf{Motivation}:
\begin{itemize}
  \item File generated from model, model got lost - > reconstruct
  \item Constraint programming is hard - is Excel hard?
  \item Avoid manual analysis, provide selection of constraints
  \item Error checking
  \item Completion, gain speed and insights (Complicated constraints, also complicated to verify, too much output)
\end{itemize}

\textbf{Novelty:}
\begin{itemize}
  \item Unsupervised setting (contrary to flashfill, etc)
  \item Numeric, different constraints (contrary to single textual function solution in flashfill, etc)
  \item Data format (2D) -- data is no longer in rows like a classic ML or DM settings
  \item Declarative, general / modular, stacking of constraint problems 
\end{itemize}

\sergey{we need structure here} 

\textbf{Approach}
\begin{itemize}
  \item Notation
  \item Algorithm (select constraints, find assignments, find solutions)
\end{itemize}

{\bfseries 
  Experimental questions
}
\begin{itemize}
  \item  How accurate are we? (Accuracy / recall)
  \item  How fast are we and which factors affect the runtime (how)?
  \item  How general is our approach, what limitations are there?
\end{itemize}


\section{Related Work}
\sergey{key bullet points for Luc and possibly Samuel and me to make related work section}

\sergey{ECAI reference style file ignores their guideline and their guideline ignores what is written in the guidelines!}
flashfill, flashextract, flashmeta \cite{flashfill,flashextract,flashmeta}
\begin{itemize}
  \item their supervised vs our unsupervised approach
  \item they look for a single ``smallest'' solution, we enumerate them all
  \item they are looking for a function, we solve constraint satisfaction problems
  \item we do not assume classic row based data layout, we work in the tabular setting
\end{itemize}

sketch \cite{sketch}
\begin{itemize}
  \item look for a constant that would fill in the gap in a program
  \item tailored for programming languages
  \item similar to model checking
  \item looks for a single solution
  \item similar to constraint satisfaction and sat, where one is interested in a single assignment that works for any potential input
\end{itemize}

tabular \cite{tabular}
\begin{itemize}
  \item language based on the excel tables that specify probabilistic models
  \item a system for probabilistic inference and similarity mostly in the usage of excel
  \item probabilistic constraint satisfaction (?) and graphical models
  \item single solution again
\end{itemize}

modelseeker \cite{modelseeker} \sergey{Samuel, Luc, probably you would need elaborate here more in details}

\begin{itemize}
  \item not designed for excel-like data representation (type consistency, groups, etc)
  \item not designed for excel-like constraints (lookups, conditional ifs, etc)
  \item does not support user extensions (?)
\end{itemize}

claudien \cite{claudien} \sergey{Samuel, Luc, you would need to help with this one}

\bibliographystyle{ecai}
\bibliography{references}
\end{document}
%%%%%%%%%%%%%%%%%%%%%%%%%%%%%%%%%%%%%%%%%%%%%%%%%%%%%%%%%%%%%%%%%%%%%%
