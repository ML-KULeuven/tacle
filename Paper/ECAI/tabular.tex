\documentclass{sig-alternate-05-2015}


\usepackage{times}
\usepackage{graphicx}
\usepackage{latexsym}
\usepackage{xspace}
\usepackage{hyperref}
\usepackage{amssymb}
\usepackage{algorithm}
\usepackage[noend]{algpseudocode}
\usepackage[numbers]{natbib}
\usepackage{notoccite}
\usepackage{framed}
\usepackage{amsmath}
\usepackage{tabularx}

\usepackage[dvipsnames]{xcolor}
\newcommand{\sergey}[1]{\textcolor{magenta}{{\sc Sergey:} #1}\xspace}
\newcommand{\samuel}[1]{\textcolor{green}{{\sc Samuel:} #1}\xspace}
\newcommand{\tias}[1]{\textcolor{blue}{{\sc Tias:} #1}\xspace}

\newcommand{\constraints}{\ensuremath{\mathcal{T}}\xspace}
\newcommand{\format}[1]{\textit{#1}\xspace}
\newcommand{\generategroups}{\format{generateAssignments}}
\newcommand{\extractgroups}{\format{extractGroups}}
\newcommand{\extracttables}{\format{extractTables}}
\newcommand{\learnconstraints}{\format{learnConstraints}}
\newcommand{\findassignment}{\format{findSolutions}}
\newcommand{\postprocess}{\format{pruneRedundant}}
\newcommand{\constrainttorder}{\format{generalityOrder}}
\newcommand{\template}{\format{constraint template}}


\newcommand{\CName}{Syntax\xspace}
\newcommand{\CSignature}{Signature\xspace}
\newcommand{\CFunction}{Definition\xspace}
\newcommand{\dependencies}{\ensuremath{\mathcal{D}}\xspace}
\newcommand{\groups}{\ensuremath{\mathcal{G}}\xspace}

\newcommand{\range}[3]{\ensuremath{#1[#2,#3]}}
\newcommand{\rangeto}[2]{#1{:}#2}
\newcommand{\rangeall}{:}

\newcommand{\eccalc}[2]{\ensuremath{#1 = #2}}
\newcommand{\ecrank}[2]{\eccalc{#1}{\mathit{RANK}(#2)}}
\newcommand{\ecfkey}[2]{\ensuremath{#1 \rightarrow #2}}
\newcommand{\ecalldiff}[1]{\ensuremath{\mathit{ALLDIFFERENT}(#1)}}
\newcommand{\eclookupf}[4]{\ensuremath{\mathit{LOOKUP}_{\mathit{#4}}(#1, #2, #3)}}
\newcommand{\eclookup}[4]{\eccalc{#1}{\eclookupf{#2}{#3}{#4}{}}}
\newcommand{\eclookupprod}[5]{\eccalc{#1}{#2 \times \eclookupf{#3}{#4}{#5}{}}}
\newcommand{\eclookupfuzzy}[4]{\eccalc{#1}{\eclookupf{#2}{#3}{#4}{fuzzy}}}
\newcommand{\ecperm}[1]{\ensuremath{\mathit{PERMUTATION}(#1)}}
\newcommand{\ecseries}[1]{\ensuremath{\mathit{SERIES}(#1)}}
\newcommand{\ecprod}[3]{\eccalc{#1}{#2 \times #3}}
\newcommand{\ectotal}[3]{\eccalc{#1}{\mathit{PREV}(#1) + #2 - #3}}
\newcommand{\ecproj}[2]{\eccalc{#1}{\mathit{PROJECT}(#2)}}
\newcommand{\ecaggc}[3]{\eccalc{#2}{\mathit{#1}(#3, col)}}
\newcommand{\ecaggr}[3]{\eccalc{#2}{\mathit{#1}(#3, row)}}
\newcommand{\ecsumc}[2]{\eccalc{#1}{\mathit{SUM}(#2, col)}}
\newcommand{\ecsumr}[2]{\eccalc{#1}{\mathit{SUM}(#2, row)}}
\newcommand{\ecaggif}[5]{\eccalc{#2}{\mathit{#1IF}(#3, #4, #5)}}
\newcommand{\ecsumif}[4]{\eccalc{#1}{\mathit{SUMIF}(#2, #3, #4)}}
\newcommand{\ecsumprod}[3]{\eccalc{#1}{\mathit{SUMPRODUCT}(#2, #3)}}

\newcommand{\numeric}{\format{numeric}}
\newcommand{\textual}{\format{textual}}
\newcommand{\integer}{\format{integer}}
\newcommand{\discrete}{\format{discrete}}
\newcommand{\plength}{\format{length}}
\newcommand{\ptype}{\format{type}}
\newcommand{\ptable}{\format{table}}
\newcommand{\por}{\format{orientation}}
\newcommand{\prows}{\format{rows}}
\newcommand{\pcols}{\format{columns}}
\newcommand{\nat}{\mathcal{N}}


\newcommand{\luc}[1]{{\textcolor{red}{#1}}}


\renewcommand{\arraystretch}{1.5}

%\usepackage{amsthm} % incompatible with ACM
%\theoremstyle{definition}
%\newtheorem{definition}{Definition}
\newdef{definition}{Definition} % ACM specific


%%\ecaisubmission   % inserts page numbers. Use only for submission of paper.
                  % Do NOT use for camera-ready version of paper.

\begin{document}

\title{Learning constraints in tabular data}
% see http://www.acm.org/binaries/content/assets/publications/article-templates/sig-alternate-sample.tex

%\author{Name1 Surname1 \and Name2 Surname2 \and Name3 Surname3 \institute{----------------------} }

\maketitle

\begin{abstract}
%Spreadsheets, comma separated value files and other tabular data representations are in wide use today. However, the correct use and maintenance of functions in and between the tabular data can be error-prone and overwhelming. In this work, we investigate the automatic discovery of constraints (functions and relations) from raw tabular data. Our method takes inspiration form inductive logic programming and constraint satisfiability. We represent common spreadsheet functions as predicates with arguments that must satisfy number of constraints over the arguments. This allows generic constraint satisfaction techniques to be used to find all such satisfying predicates/functions within the rows, columns and submatrices of the tables. We show the effectiveness and accuracy of this approach on a number of spreadsheets from varying sources.
Spreadsheets, CSV (comma separated value) files and other tabular data representations are in wide use today. However, modeling, maintaining and discovering formulas in tabular data and between spreadsheets can be time consuming and error-prone. In this work, we investigate the automatic discovery of constraints (functions and relations) from raw tabular data. We see multiple promising applications for this technique, e.g. in rediscovering constraints, auto-completion and error checking. Our method takes inspiration from inductive logic programming, constraint learning and constraint satisfaction. Common spreadsheet functions are represented as predicates whose arguments must satisfy a number of constraints. Constraint satisfaction techniques are used to identify predicates (constraints) that hold between the rows, columns and submatrices of the tables. Therefore, new types of constraints can be easily added to the system by specifying them using declarative methods such as Minizinc or ASP. We show that our approach is able to accurately discover constraints in a number of spreadsheets from varying sources.
\end{abstract}

\section{Introduction}
Millions of people across the world use spreadsheets everyday. The tabular representation of the data is often intuitive, and the programming of functions in individual cells is quickly learned. However, large sheets (possible with multiple tables) and relations between different tables and sheets can be tedious and error-prone to handle. As a result, many people lack a complete understanding of the structures and existing dependencies in the data.

This limited understanding is especially the case with spreadsheets exported from other software such as ERP packages. In this case, often a comma-separated values (CSV) format is used, meaning that all formula's are lost, including inter-sheet formulas and relations.

In this work, we investigate in how far machine learning techniques can be used to infer constraints (formula's and other relations) from raw spreadsheet data. However, this is an unconventional machine learning problem. 
Consider the example in Figure~\ref{fig:main_example}, where the header's names already suggest the usage of spreadsheet operations such as \textit{average} or \textit{sum}.
Looking at the first row in Table~2 and the data in Table~1, it is clear that the values in this row were computed by summing the values in the corresponding column of Table~1.
Examining Table~1 also shows that computations are not only performed column-wise but row-wise as well: the cells in the column \textbf{Total} are obtained by taking the row-wise sum of the previous four columns.
This already provides a flavor of how this problem is different from standard data mining settings, when the data is just in rows and variables are in columns. Here everything is mixed. The data is relational on the one hand, since we have multiple table with relationships between them. And on the other hand the data is mixed textual and numeric, since people tend to use spreadsheets for financial and accounting computations. One can see that understanding and learning constraints in tabular data is hence a new and challenging problem setting.
%Furthermore, it is not only challenging for a machine but for a human as well.

Relatedly, the field of Inductive Logic Programming has long... \tias{TODO for Sergey or Luc}. Additionally, the field of Constraint Programming has recently started investigating ways in which Constraint Satisfaction Problems can automatically be inferred from known solutions. Here, methods ranging from generate-and-test~\cite{todo:modelseeker}, concept learning~\cite{todo:conacq} and Inductive Logic Programming~\cite{todo:lallouet} have been used with encouraging results.

Another line of work from which we draw inspiration is that of \tias{TODO for ?, flashfill et al}. We will consider the case where all data is given at once. However, as in the use case of flashfill, one could also use this technique to derive formula's interactively while a user is entering values in a cell \tias{TODO SAMUEL: proofread/refine last statement}.

The question that we hence try to answer in this work is: can we discover or reconstruct structural constraints (relations, functions) in flat tabular spreadsheet data? Moreover, we investigate general-purpose ways for doing so, that do not require hard coding a number of constraints. Rather, we use techniques from constraint solving to derive whether and which constraints are satisfied. This approach can be extended to any constraint in a principled way.



%Due to the complexity of numeric computations in spreadsheets, people often fail to grasp the properties of the data, which leads to the so called Spreadsheet risk\footnote{\url{https://en.wikipedia.org/wiki/Spreadsheet\#Spreadsheet_risk}} and even caused major flaw in the famous economics papers \cite{flaw_excel} and billion-losses in the financial industry \cite{spreadsheet_risk_loss}. Tabular constraint learning provides a potential cure by indicating learned constraint violations and suggest possible fixes.

{\it
\textbf{Motivation}:
\begin{itemize}
  \item USED -- File generated from model, model got lost, need to reconstruct
  \item Constraint programming is hard - is Excel hard?
  \item Avoid manual analysis, provide selection of constraints
  \item SOMEWHAT USED --Error checking
  \item Completion, gain speed and insights (Complicated constraints, also complicated to verify, too much output)
\end{itemize}

\textbf{Novelty:}
\begin{itemize}
  \item USED -- Unsupervised setting (contrary to flashfill, etc)
  \item Numeric, different constraints (contrary to single textual function solution in flashfill, etc)
  \item USED -- Data format (2D) -- data is no longer in rows like a classic ML or DM settings
  \item USED -- Declarative, general / modular, stacking of constraint problems
\end{itemize}

\begin{figure*}[tbh]
  \begin{center}
    \includegraphics[width=0.75\textwidth]{figures/Demo.png}
  \end{center}
  \vspace{-10pt}
  \caption{Example spreadsheet (black words and numbers only). Blue and green elements are added by us to clarify the relations.}
  \label{fig:main_example}
\end{figure*}
}

\section{Formalization}
In this work we focus on spreadsheet data. We will first introduce the general concepts of cells, group and constraint templates, using the (multi-table) sheet in Figure~\ref{fig:main_example} as running example.

\subsection{Cells, vectors and groups}
A spreadsheet is physically one big table, but conceptually may consist of multiple tables, such as in Figure~\ref{fig:main_example}. We will describe a method of extracting tables from a big spreadsheet in Section~\ref{todo}, however this is not the core of our learning method. Our formalization reasons over tables directly.

Formally, a table is an $n \times m$ matrix. Each entry is called a \textit{cell}.
A cell has a {\em type}, which can be numeric or textual. We further distinguish numeric types in subtypes integer and float. We also consider \textit{None} as a special type when a cell is empty; \textit{None} is a subtype of all other types.
A set of cells is called \textit{type-consistent} iff all cells in the set are of the same type.
%Certain constraints, such as \textit{rank} or \textit{series}, make use of the numeric subtype by requiring its arguments to be integers.

We consider various sets of cells with the following notation:
\begin{itemize}
  \item {\bf Cell}: We use the notation $T[a,b]$ to refer to a particular cell in the table~$T$ at the row with index~$a$ and the column with index~$b$.

  \item {\bf Columns and rows}: For a table $T$, we use notation $T[a,{:}]$ to refer to the $a$-th row, and similarly $T[{:},a]$ for the $a$-th column. % to refer to a subrange from $a$ to $b$ we use $T[a{:}b,]$ ($T[,a{:}b]$).

  \item
  A \textbf{vector} is a type-consistent row or column.
  If a vector is a row (column), we say that it has a \textit{row} (\textit{column}) orientation. The \textit{length} of a vector is the number of cells in that vector.

  \item
  A \textbf{group} is a projection of a matrix onto a set of consecutive rows or columns. %In this work we only consider consecutive rows and columns.
  We use the following notation to refer to a row-oriented group with rows ranging from $a$ to $b$: $G = T[a{:}b,:]$, where $a,b$ are natural numbers. Similarly, we have $G = T[{:},a{:}b]$ for a column-oriented group.
  The vectors in a group have to be type-consistent, consecutive in the original table and of equal length.
  The \textit{size} of a group is the number of vectors in the group, while the \textit{length} of a group is the length of its vectors. 
  %The \textit{length} of a row (column) group $G$, written as \textit{length(G)}, is the number of its columns (rows); We call a group $G$ \textit{numeric} (\textit{textual}, etc), written as \textit{numeric(G)}, if all its vectors contain numeric (textual, etc) elements.

  \item
  A \textbf{subgroup} $G'$ of $G$ is simply a projection of $G$ onto a consecutive subset its vectors. % is a subrange of vectors belonging to a group $G$. If the subgroup $g$ must contain only one vector, we write $g \in G$. %It defines what it means for a constraint to be well-formed
%We refer to a subgroup of a group $G$ in \groups as $g$. A \textit{subgroup} $g$ is a subrange of vectors in a group $G$. If a subgroup $g$ must contain only one vector, we write $g \in G$.
% A \textit{subgroup} $G'$ is a subrange of vectors belonging to a group $G$. If the subgroup $\{v\}$ must contain only one vector, we write $v \in G$.
\end{itemize}
%For notational convenience, we will refer to vectors in {\rm roman} and to groups in {\bf bold}. {\bf CHECK}

%When we do not want to distinguish between cells, columns, rows, vectors or groups, we shall talk about a {\em set of cells}.

\paragraph{Example}
Consider Table~1 in the Figure~\ref{fig:main_example}, its rows are not type consistent (i.e. they contain both numeric and textual data).
Ignoring the (all-textual) header, Table~1 can be divided into the following five column-oriented groups:
{\small
\begin{align*}
&G_1 = \range{T_1}{\rangeall}{1},
&G_2 = \range{T_1}{\rangeall}{2},\\
&G_3 = \range{T_1}{\rangeall}{\rangeto{3}{8}},
&G_4 = \range{T_1}{\rangeall}{9},\\
&G_5 = \range{T_1}{\rangeall}{\rangeto{10}{11}}.
\end{align*}
}
An example subgroup is $\range{T_1}{\rangeall}{\rangeto{3}{6}} \subset G_3$, which contains the sales numbers of all employees for the four quarters.

\subsection{Constraint templates}
Our method will extract the applicable constraints (including functions) in a set of tables, based on \textit{constraint templates}.
%
A \template is a triple \textit{(\CName, \CSignature, \CFunction)}:
%Let us elaborate on this:
\begin{itemize}
\item
\textit{\CName}  specifies the syntactic form of the constraint $c(A_1, ...,A_n)$, that is, the name of the constraint $c$ together
with $n$ arguments $A_i$. Thus a constraint $c$ is viewed as a relation or predicate of arity $n$ in first order logic. Note that a function $B=f(A_1,...,A_n)$ can be represented with the $n+1$-ary predicate $cf(B,A_1,...,A_n)$.
% the list of its variables $v_1,\dots,v_n$.  In ILP terminology, it is known as vocabulary.

\item the \textit{\CSignature} defines the properties that the arguments of the predicate must satisfy. That is, the required types of the arguments, together
with possible further restrictions such as on their length. For example, the signature can restrict the types of the arguments to be groups integers and require that the length of all arguments be equal. \\
In terms of logical and relational learning, such a \CSignature is known as the {\em bias} of the learner, it specifies when the constraint is well-formed.

\item \textit{\CFunction} is the actual definition of the constraint that specifies when the constraint holds. Given an assignments to all its arguments, it can verify whether the constraint is satisfied or not. %holds, in practice this will be a function that can be called when
%the arguments are full instantiated to decide whether the constraint is satisfied or not.
In logical and relational learning this is known as the background knowledge, which provides the definition of the predicate.
\end{itemize}

\paragraph{Example}
Several example constraints are illustrated in Table 1.  For example, the constraint template \textit{rank} has
\begin{itemize}
\item \CName: \textit{$B$ = RANK($A$)}, where $B$ and $A$ are the arguments;
\item \CSignature: the two arguments $B$ and $A$ must be numeric groups and of size 1; furthermore, the group $B$ must be integer (the rank) and the length of groups $A$ and $B$ must be identical, i.e., $length(B) = length(A)$;
\item \CFunction: \textit{$B$ = RANK($A$)} whenever each value of $B$ is the rank of the corresponding value in $A$.
% is satisfied in the mapping $\{ v_x \mapsto g_x, v_y \mapsto g_y \}$ satisfies the constraint iff $g_x \in G_x, g_y \in G_y$ ($G_x \in \groups, G_y \in \groups$) and each value in $g_y$ is the rank of the corresponding value in $g_x$ (possibly with ties).
\end{itemize}

When looking into constraints like \textit{$B$ = RANK($A$)}, it is helpful to see the analogy with both first order logic (FOL) and with constraint satisfaction problems.
From a FOL perspective, the name of the constraint ($RANK$) is just the predicate, and the arguments $B$ and $A$ are the terms, which can be seen as either uninstantiated variables or as values (concrete groups with values).
This also holds in our setting: when we write
      $\ecrank{\range{T}{\rangeall}{8}}{\range{T}{\rangeall}{7}}$,
we can interpret the argument ${\range{T}{\rangeall}{8}}$ as a group variable that would apply to any table \tias{Of the right (minimum) dimension?? Note that this is also sloppy because the ':8' is not considered part of the variable...},
or we can interpret the argument as the value it takes in a particular table $T$ such as $T_1$ in Figure 1, that is, $[1,2,3,4]$.
This also means that we can talk about constraints being satisfied. Basically, a constraint $c(A_1, ..., A_n)$ is {\em satisfied}  in a set of tables ${\cal T}$ if and only
if the constraint holds on the values that the $A_i$ take in the set of tables ${\cal T}$. \tias{I find this confusing: a group as we defined it is defined on one table, not on multiple tables. we should separate the table variable from the 'range' variables then...}
%Figure 3 provides a list of constraints that are satisfied by the tables in Figure 1.



%\sergey{moved from the formalization BEGIN}
%l 1) a set of cells, and 2) a set of constraints over those cells.  Notice that the constraints specify the types (or the domains) that one encounters in traditional CSPs.  Notice also, that as in traditional CSPs, the CSP problem defines the variables (the cells) of the problem as well as the constraint that should hold amongst them. The task of CSP-solvers is then to find an assignment to the variables that satisfies all the constraints.  The distinction between variables (cells) and their assignments (their values) is important here.  So, when we speak about cells, vectors and groups, we refer to the variables; when we talk about values, assignments or instances, we refer to the values these cells take in a particular "instantiated" table.

\section{Problem Statement}
The problem of learning constraints from tabular data can be seen as an inverse {\em constraint satisfaction problem} (CSP). A CSP consists of a set of constraints and the
task is to find assignments of values to the variables in the CSP so that all constraints are satisfied.
In the context of spreadsheets where variables could represent cells in tables, solving a corresponding CSP problem would mean completing the tables so that all constraints of the spreadsheet are satisfied.

The inverse problem is, given an assignment to all cells, to find the constraints that are present in the spreadsheet.
We define the {\bf Tabular Constraint Learning Problem} as follows:
%
\begin{definition} \textit{Tabular Constraint Learning.}\label{def:problem_statement}\\
  Given ${\cal G}$ a set of groups over fully instantiated tables ${\cal T}$ (all cells have values) and a set of {\template}s ${\cal S}$: find the set of all constraints $s(G_1', ..., G_n')$ that match the syntax of a \template~$s \in {\cal S}$ and where the (sub)groups $G_1', ... , G_n'$ with $\forall i: G_i' \subseteq G_i, G_i \in {\cal G}$ satisfy both the signature and the definition of that \template.
\end{definition}
% {\bf Given }
% \begin{enumerate}
% \item
% a set of instantiated tables (so all cells have values) ${\cal T}$;
% \item
% a set of {\template}s ${\cal C}$;
% \item
% a set of groups ${\cal G}$ for the tables ${\cal T}$;
% \end{enumerate}
% \noindent
% {\bf Find}  the set of all constraints $s(G_1', ..., G_n')$ that 
% \begin{enumerate}
% \item are satisfied  in ${\cal T}$;
% \item for which all $G_i' \subseteq G_i \in {\cal G}$ ;
% \item the groups $G_1', ... , G_n'$ satisfy the signature of the constraint template~$s \in {\cal C}$.
% \end{enumerate}
An example solution for the type-consistent groups of the tables in Figure~\ref{fig:main_example} and {\template}'s in Table~\ref{table:constraints} is shown in Figure~\ref{fig:sol_example}.

% Let us introduce \template instantiation syntax. Let $S$ be a mapping from variables to the subgroups of \groups and $t$ be a \template, then $t^S$ is the template, in which all variables in all constrainst in \CSignature and \CFunction are instantiated with the corresponding subgroups from $S$. We refer to $t^S$ as a \textit{constraint} and to the instantiated \CSignature (\CFunction) as $\text{\CSignature}^S$ ($\text{\CFunction}^S$). Throughout the work we refer as \textit{find constraints} to finding all constraints that hold.


% Let us now define what it means for a constraint from \template to hold in general. Let $t$ be a \template, $S$ be a mapping from variables to the subgroups of \groups, and $C$ be the constraint $t^S$, then $C$ \textit{holds} iff all constraints in $\text{\CSignature}^S$ and $\text{\CFunction}^S$ of $C$ hold.
% \sergey{moved from formalization END}

% In the previous section we introduced the problem of tabular constraint learning informally using the example in Figure \ref{fig:main_example}. Here we formalize the statement in terms of \template and group assignments as follows:

% \begin{minipage}[c]{14em}
%   \vspace{5pt}
%   \begin{tabular}{ll}
%     \multicolumn{2}{l}{{\textbf{Tabular Constraint Learning Problem}}}\\
%     \vspace{-4pt}
%     &\\
%     \textbf{Given:}& the set of all groups $\groups$ and of \template $\constraints$\\
%     \textbf{Find:}&  all constraints $C$ over \groups for each template $t$ in \constraints \\
%   \end{tabular}
%   \vspace{6pt}
% \end{minipage}
% \luc{To here}

As is well-studied in Inductive Logic Programming, some constraints can \textit{entail} other constraints, meaning that some constraints are \textit{implied} by others. More formally, a constraint $c_1$ implies another constraint~$c_2$ iff whenever $c_1$ is true, $c_2$ must also be true. As a result, a solution to the tabular constraint learning problem may contain constraints that are implied by others. We can hence say that there is \textit{redundancy} in the solution set of constraints.


Two types of impliciation:

1) on constraint definitions for same signature (example: permutation and alldiff or foreign-key and alldiff)

2) on constraint signature for same definition (example: ?) \textit{Formally, we say that a constraint template $t_1$ is \textit{more specific} than a constraint template $t_2$ iff for $t_1$ and $t_2$  holds the statements of the form: $\forall g_1, \dots, g_n{:}~ t_1(g_1,\dots,g_n) \rightarrow t_2(g_h, \dots, g_k)$ where $1 \leq h \leq k \leq n$. The intuition behind the formula is the following, if a list of groups is a solution for $t_1$ then its subset is also a solution for $t_2$.}

3) combination

\tias{OLD TEXT:}

A constraint template~$s_1$ is more specific than a template~$s_2$ iff a constraint applying $s_1$ to a set of subgroups is always more specific than a constraint applying $s_2$ to (part of) the same set.
\samuel{Formalize: $\forall x1..x_k s_1(x_1...x_n) => s_2(x_1...x_k)$ :: looking for most specific}

\paragraph{Example}
Consider templates \textit{all-different} and \textit{foreign-key}.
If \ecfkey{G_{pk}}{G_{fk}}, then \ecalldiff{G_{pk}} must also hold, therefore, \textit{all-different} is considered to be more general than \textit{foreign-key}.
\\\\
In ILP one often exploits these type of generality relations to structure the search for constraints using refinement operators \cite{luc_book}.
Using such techniques leads to a more efficient traversal of the search space (of constraints).
Aside from the efficiency gained by exploiting constraints learned earlier, the generality also allows us to build constraint templates on top of more general templates.

\tias{End old text}

In theory each constraint is independent and equally useful. In practice, however, things are different and some constraints are considered to be \textit{redundant}. Assume that for some $g_x$ both \textit{series($g_x$)} and \textit{alldifferent($g_x$)} hold. From a user's perspective the last solution is useless, since he or she already knows that the series constraint implies the alldifferent constraint, or formally speaking $\forall g_x{:}~\textit{series}(g_x) \rightarrow \textit{alldifferent}(g_x)$  holds. To capture this condensed representation \cite{condensed} of solutions, we have introduced a specificity order between constraints, which is depicted using solid arrows in Figure \ref{fig:learning_order}. This matters from implementation perspective as well, if we discovered that $g_x$ is not a permutation, then $g_x$ should not be tested for the series constraint at all. 


Formally, we say that a constraint template $t_1$ is \textit{more specific} than a constraint template $t_2$ iff for $t_1$ and $t_2$  holds the statements of the form: $\forall g_1, \dots, g_n{:}~ t_1(g_1,\dots,g_n) \rightarrow t_2(g_h, \dots, g_k)$ where $1 \leq h \leq k \leq n$. The intuition behind the formula is the following, if a list of groups is a solution for $t_1$ then its subset is also a solution for $t_2$.


Let us introduce the version of tabular constraint learning by modifying Definition \ref{def:problem_statement} that takes into account the specificity relation between constraint templates.

\begin{definition} \textit{Condensed Tabular Constraint Learning}\label{def:condensed_problem_statement}\\
  Given a set of instantiated tables~${\cal T}$ (all cells have values), a set of {\template}s~${\cal S}$, a set of groups~${\cal G}$ over the tables~${\cal T}$, and the set of all specifity rules~\dependencies: find the set of all constraints $s(G_1', ..., G_n')$ that are satisfied  in ${\cal T}$, for which all $G_i' \subseteq G_i \in {\cal G}$, the (sub)groups $G_1', ... , G_n'$ satisfy the signature of the constraint template~$s \in {\cal C}$ and there is no $s'$ in ${\cal C}$ such that $s'(G_h',\dots,G_k')$ holds (for $1 \leq h \leq k \leq n$) and  $s'$ is more specific than $s$ in \dependencies.
\end{definition}


% \begin{minipage}[c]{14em}
%   \vspace{5pt}
%   \begin{tabular}{ll}
%     \multicolumn{2}{l}{{\textbf{Condensed Tabular Constraint Learning Problem}}}\\
%     \vspace{-4pt}
%     &\\
%     \textbf{Given:}& the set of all groups $\groups$ and of \template's $\constraints$,\\
%     & a constraint specificity DAG \dependencies \\
%     \textbf{Find:}& all the most specific wrt \dependencies constraints $C$ over \groups\\
%     & for each $t$ in \constraints \\
%   \end{tabular}
%   \vspace{6pt}
% \end{minipage}

This version explicitly takes into account the specificity of the constraints and invalidates some of the solutions that are entailed by the tabular other solutions.

\sergey{still need to remove some repetitions later on about specisifity}

\newcommand{\tcl}{Tabular Constraint Learning}
\newcommand{\ctl}{Condensed Tabular Constraint Learning}
\section{Approach to Tabular Constraint Learning}
% \sergey{Luc wants this to be more intuitive, should we just add some examples here to make it easier to follow and refer to the running example?}
This section will describe our approach to \tcl.

The key observation is that essentially the problem is a constraint enumeration problem, where each constraint is independent. This property comes from fact that in spreadsheets each formula is applied based on the existing data in the cells. This allows learning constraints independently of each other by examining constraint satisfaction on the groups.

Our algorithm reasons on the level of groups and vectors on the data and assumes that these are provided.
Often, however, this not the case and a possible approach is to extract the groups from the CSV file directly.
This requires discovering \textit{tables} and \textit{groups}.
Our implementation offers automated procedures for both steps.



\subsection{Table and group detection}
\samuel{Is an example needed? I find no because it is not our main concern but open for discussion}
We chose to implement a naive approach to extract tables as the detection of tables and headers from data is a complex task on its own (\cite{header}) and a sophisticated implementation goes beyond the scope of this paper.
Our approach first detects rectangular collections of data.
It then attempts to identify the header by looking at the type of the first row.
If this row consists only of textual entries, the row is expected to be the header and is removed, if not, the same procedure is applied on the first column.
We suggest to provide tables in general, for well formated data, however, the user will not have to specify tables manually.

Our approach for detecting groups works by partitioning the data of a table into consecutive, type consistent groups, essentially separating textual and numerical data.
The assumption is that often numerical and textual data are not mixed and this approach works well in practice as often groups do not have to be specified manually\footnote{\samuel{Check on benchmark}}.
Unless an explicit orientation (rows or columns) is provided for the table, this approach will attempt to create both row- and column-groups.

So far we have considered textual and numerical (floating point or integer) data.
Our implementation also supports some extensions such as currencies and percentual data, which is considered to be numerical data.
Moreover, percentual data is corrected to the floating point representation (i.e. divided by 100).

A more elaborate implementation could allow group (and table) detection to be done interactively / visually such that a user can select the groups in a spreadsheet or correct automatically generated groups.
From here on we will assume that groups and tables are given.

\newcommand{\temps}{\ensuremath{S}}

\subsection{Algorithm}
Our algorithm is illustrated in Algorithm~\ref{algo:tcl} and contains three key steps.
First, constraint templates are ordered according to the generality ordering such that they can be treated sequentially.
Then, for each constraint template~$s \in \temps$ we generate a set~$A$ of \textit{group-assignments} that satisfy the \CSignature of~$s$.
Every group assignment assigns a group~$G \in \groups$ to every argument of the template.
In the third step, for every assignments $(G_1, ..., G_n) \in A$ we find all constraints $s(G_1', ... G_n')$ that hold on the data (satisfy the \CFunction), such that every subgroup $G_i' \subseteq G_i$.

\begin{algorithm}[thb]
  \begin{algorithmic}
    \footnotesize
    \State \textbf{Input:} $\groups$ -- groups, $\constraints$ -- \template's, \dependencies -- dependency graph
    \State \textbf{Output:} $C$ -- learned constraints
    \State $C \gets \emptyset$ \Comment{The set of constraints}
  \For{$t \in \constrainttorder(\constraints,\dependencies)$}
  \State $v_1, ..., v_n =$ variables of $T$
    \For{$v_1{:}~G_1, \dots, v_n{:}~G_n \in \generategroups(t, \groups, C, \dependencies)$}
      \State $C \gets C \cup \findassignment(t, v_1{:}~G_1, \dots, v_n{:}~G_n, C, \dependencies)$
    \EndFor
  \EndFor\\
\Return $C$
\end{algorithmic}
\caption{Tabular constraint learning}
\label{algo:tcl}
\end{algorithm}


Our Algorithm \ref{algo:tcl} is in line with the ``generate-and-test'' paradigm that is well-known in AI \cite{whaisasp}, however, we apply the approach on two levels, generating and testing groups according to their properties and generating and testing subgroups on the actual data.
We will explain and illustrate the three key steps in detail.

\subsubsection{Generality ordering}
\tias{ordering is because some constraint templates use other templates in their signature, doesn't have to be more complicated?}
\samuel{That is true - is it clear now?}

\begin{figure}[tbh]
  \centering
  \includegraphics[width=0.9\linewidth]{figures/constraint_dependency.png}
  \caption{Constraint Learning Order.
  An arrow from~$s_1$ to~$s_2$ indicates that the \CSignature of~$s_2$ requires some arguments of~$s_2$ to fulfill~$s_1$.
  Therefore, constraints for the more general \template~$s_1$ are learned before constraints for the more specific \template~$s_2$.
  }
  \label{fig:learning_order}
\end{figure}

$\constrainttorder(\constraints,\dependencies)$ uses the DAG~\dependencies in Figure \ref{fig:learning_order} to generate all constraint templates according to the topological order of~\dependencies.
Constraint templates that have no dependencies are omitted.

\paragraph{Example}
As mentioned earlier, \textit{foreign-key} is more specific than \textit{all-different} as $\ecfkey{v_{fk}}{v_{pk}} \implies \ecalldiff{v_{pk}}$.
Therefore, we can first generate all subgroups $G_i'$ for which \ecalldiff{G_i'} is satisfied and use these subgroups as candidates for $v_{pk}$ in \ecfkey{v_{fk}}{v_{pk}}.

The \ecrank{v_y}{v_x} constraint does not have any dependencies, however in a restricted version without ties, groups assigned to $v_y$ would always be a permutation of numbers $1, 2, ..., \textit{length}(v_y)$.
\\\\
Most parts of the algorithm could be easily parallelized.
Every connected component in \dependencies can be learned independently.
Within every component, constraint templates have to be treated respecting the order, i.e. learning the most general ones first.
The generality ordering thus imposes a partial ordering according to which constraint templates must be learned.

\subsubsection{Candidate group generation}
$\generategroups(\textit{Constraint,GroupSet,Solutions})$ is the function generating tuples of groups (group-assignments~$A$) that may contains subgroups satisfying the constraint.
The goal of this step is to prune out impossible groups from the beginning by looking at their properties rather than their actual content (\CSignature vs \CFunction).

Finding candidate groups for a \template~$t$ can be seen as a Constraint Satisfaction Problem (CSP) dependent on the \CSignature of~$t$.
For every argument in~$t$ we consider all groups~\groups (or all candidates from a more general constraint) and prune these candidate groups using their types and other requirements imposed by the templates \CSignature, such that a group~$G \in \groups$ is selected if there is at least one subgroup~$G' \subseteq G$ that satisfies the \CSignature.

\samuel{TODO: correct code integer groups}\\
Consider the \ecrank{v_r}{v_x} constraint template, then both arguments must be of the same length.
Therefore, assignments containing groups of different length do not have to be considered.
Constraints of the \CSignature cannot always be enforced directly on a group but have to consider whether subgroups could satisfy them.
For example, the \CSignature requires $v_r$ to be integer.
If a numeric group is assigned to $v_r$ then some subgroups may have type float while other may be of type integer, therefore, this this group cannot be discarted.

For some constraints (e.g. sum, max, count) that share the same \CSignature the same set of assignments~$A$ will be found.

\paragraph{Example}
\ecrank{v_r}{v_x} will require groups assigned to $v_r$ to be $\integer$, groups assigned to $v_x$ to be $\discrete$ (i.e. $\integer$ or $\textual$) and only allows assignments in which groups assigned to both arguments have the same length.

\subsubsection{Subgroup satisfaction search}
$\findassignment(t,A,C_{solutions})$ finds, for all assignments~$a = (G_1, ..., G_n)$ in $A$, the subgroups $G_i' \subseteq G_i$ such that the corresponding constraint $t(G_1', ..., G_n')$ is satisfied (i.e. the \CSignature and \CFunction of~$t$ are satisfied).
For every \template~$t$ an implementation (declarative or imperative) needs to be provided that, given an assignment, finds valid subgroups.

\paragraph{Example}
Consider a group assignment $G_x, G_y$ associated with the arguments $v_x,v_y$ of \ecrank{v_y}{v_x}, then \findassignment selects single vectors $V_y \in G_y, V_x \in G_x$ such that \ecrank{V_y}{V_x} is true.
For example, in Figure~\ref{fig:main_example} the group $G = \range{T_1}{\rangeall}{\rangeto{3}{8}}$ can serve as both $G_x$ and $G_y$.
Then \findassignment could select $V_x$ to be $\range{T_1}{\rangeall}{7}$ and $V_y$ to be $\range{T_1}{\rangeall}{8}$ since \ecrank{\range{T_1}{\rangeall}{7}}{\range{T_1}{\rangeall}{8}} is satisfied in the data.
\\\\
For some constraint templates (most notably aggregates) there may be multiple (overlapping) subgroups of a given group~$G$ for which the template satisfied.
In this case we choose to omit those subgroups that are fully contained within a larger subgroup since this can lead to over-fitting.
If the returned subsets are too large and accidentally include unrelated vectors the user can split such a group.

\subsection{Implementation note}
For each \template, both \generategroups and \findassignment are can be seen as separate constraint satisfaction problems.
Hence, it is natural to solve them using declarative solvers.
Even though each of these problems differ from each other, they are variations that can be modeled using similar techniques.
This also makes it easy to extend the implementation as templates can be easily added (or removed).
Furthermore, declarative languages such as, for example, ASP \cite{whaisasp} and Minizinc \cite{minizinc} provide primitives that make modeling transparent and generic.

Our system internally supports ASP (using Clingo 4.5.4 \cite{clingo}), Minizinc (Gecode-backend 2.0.2 \cite{minizinc}) and uses a native Python CSP solver \cite{python_constraint} for \generategroups.
All the constraints mentioned in the paper can be run purely using the internal Python engine without any external software.
Future versions of the system will provide user interfaces to some of these constraint languages (e.g. ASP and Minizinc) to allow users to specify solving procedures for new constraint templates in a declarative way.

\subsection{Constraints}

\begin{table*}
  \centering
  \begin{tabularx}{\textwidth}{l X X}
    \textbf{\CName} & \textbf{\CSignature} & \textbf{\CFunction}\\ \hline \hline
    $\ecalldiff{v_x}^*$
      & $\discrete(v_x)$
      & All values in $v_x$ are different: $v_x[i] \neq v_x[j]$ if $i \neq j$
      \\ \hline
    %X = Y & & \\
    $\ecperm{v_x}^*$
      & $\numeric(v_{x})$, $\ecalldiff{v_{x}}$
      & The values in $v_{x}$ are a permutation of the numbers $1$ through $\plength(v_{x})$.
      \\ \hline
    \ecseries{v_x}
      & $\numeric(v_{x})$ and $\ecperm{v_{x}}$
      & $v_{x}[1] = 1$ and $v_{x}[i] = v_{x}[i - 1] + 1$.
      \\ \hline
    \ecfkey{v_{fk}}{v_{pk}} & $v_{fk}$ and~$v_{pk}$ are both $\discrete$; $\ptype(v_{fk}) = \ptype(v_{pk})$; $\ptable(v_{fk}) \neq \ptable(v_{pk})$; and $\ecalldiff{v_{pk}}$ & Every value in~$v_{fk}$ also exist in~$v_{pk}$ \\ \hline
    % TODO check out
    \eclookup{v_r}{v_{fk}}{v_{pk}}{v_{val}}
      & $v_{fk}$ and $v_{pk}$ are both $\discrete$; arguments $\{v_{fk}, v_{r}\}$ and $\{v_{pk}, v_{val}\}$ within the same set have the same \plength, \ptable and \por; $v_{r}$ and~$v_{val}$ have the same type; and \ecfkey{v_{fk}}{v_{pk}}.
      & $v_r[i]$ is the same value as looking up~$v_{fk}[i]$ in~$v_{pk}$  and returning the corresponding value in~$v_{val}$.
      \\ \hline
    \eclookupfuzzy{v_r}{v_{fk}}{v_{pk}}{v_{val}}
      & \textit{Same as lookup}
      & $v_r[i]$ is the same value as looking up the last item in~$v_{pk}$ smaller than~$v_{fk}[i]$ and returning the corresponding value in~$v_{val}$.
      \\ \hline
    \eclookupprod{v_r}{v_1}{v_{fk}}{v_{pk}}{v_{val}}
      & Arguments $\{v_{r}, v_{1}, v_{fk}\}$ are $\numeric$, arguments $\{v_{pk}, v_{val}\}$ are $\discrete$ and within both sets all arguments have the same \plength, \ptable and \por; also \ecfkey{v_{fk}}{v_{pk}}.
      & $v_{r}[i]$ is the obtained by multiplying $v_{1}[i]$ with $\eclookupf{v_{fk}}{v_{pk}}{v_{val}}{}[i]$.
      \\ \hline
    \ecprod{v_r}{v_1}{v_2}
      & Arguments $\{v_{r}, v_{1}, v_{2}\}$ are all $\numeric$ and have the same $\plength$
      & $v_{r}[i] = v_{1}[i] \times v_{2}[i]$.
      \\ \hline
    \ecproj{v_r}{\mathbf{v_x}}
      & Arguments $\{v_{r}, \mathbf{v_x}\}$ all have the same $\plength$, $\por$, $\ptable$ and $\ptype$; $\mathbf{v_x}$ contains at least~2 vectors; and $v_r = \mathit{SUM}(\mathbf{v_x}, \por(\mathbf{v_x}))$
      & At every position~$i$ in $1$ through $\plength(v_{r})$ there is exactly one vector~$v$ in $\mathbf{v_x}$ such that $v[i]$ is a non-blank value, then $v[i] = v_{r}[i]$.
      \\ \hline
    \ecrank{v_r}{v_x}
      & $\integer(v_{r})$; $\numeric(v_{x})$; and $\plength(v_{r}) = \plength(v_{x})$
      & The values in $v_{r}$ represent the rank (from largest to smallest) of the values in $v_{x}$ (including ties)
      \\ \hline
    \ectotal{v_r}{v_{pos}}{v_{neg}}
      & Arguments $\{v_{r}, v_{pos}, v_{neg}\}$ are all $\numeric$ and all have the same $\plength$, which is at least $2$
      & The values in $v_{r}$ are a running total, each value $v_{r}[i] = v_{r}[i - 1] + v_{pos}[i] - v_{neg}[i]$.
      \\ \hline
    $\ecsumc{v_r}{\mathbf{v_x}}^\dagger$
      & $v_r$ and $\mathbf{v_x}$ are $\numeric$; $\prows(\mathbf{v_x}) \geq 2$; and $\pcols(\mathbf{v_x}) = \plength(v_r)$
      & Each value in $v_{r}$ is obtained by summing over the corresponding column in $\mathbf{v_x}$.
      \\ \hline
    $\ecsumr{v_r}{\mathbf{v_x}}^\dagger$
      & $v_r$ and $\mathbf{v_x}$ are $\numeric$; $\pcols(\mathbf{v_x}) \geq 2$; and $\prows(\mathbf{v_x}) = \plength(v_r)$
      & Each value in $v_{r}$ is obtained by summing over the corresponding row in $\mathbf{v_x}$.
      \\ \hline
    $\ecsumif{v_r}{v_{fk}}{v_{pk}}{v_{val}}^\dagger$
      & $v_{fk}, v_{pk}$ are $\discrete$; $v_{r}, v_{val}$ are $\numeric$; within the sets $\{v_{val}, v_{fk}\}$ and $\{v_{pk}, v_{r}\}$ arguments have the same $\plength$ and $\por$; $v_{fk}$ and $v_{val}$ have the same $\ptable$; $v_{fk}$ and $v_{pk}$ must have different $\ptable$s but the same $\ptype$; and \ecalldiff{v_{pk}}
      & The value for $v_{r}[i]$ is obtained by summing all values $v_{val}[j]$ where $v_{fk}[j] = v_{pk}[i]$
      \\ \hline
    \ecsumprod{v_r}{v_1}{v_2}
      & Arguments $\{v_r, v_1, v_2\}$ are all $\numeric$; $\plength(v_{1}) = \plength(v_{2}) \geq 2$; and $\prows(v_{r}) = \pcols(v_{r}) = 1$
      & $v_{r}[i] = \sum_{i = 1}^{\plength(v_{1})} v_{1}[i] \times v_{2}[i]$.
      \\


  \end{tabularx}
  \caption{
An overview of the different constraint templates that we have implemented.
Most of the constraint correspond directly to spreadsheet functions and have been selected based on the most popular constraints in Excel and their occurence in spreadsheets that we have collected.
The constraints marked with $*$ are \textit{structural}, they are discovered by our algorithm, however, they cannot be readily used in spreadsheet software.
The constraints marked with $\dagger$ are aggregate constraints, the table shows the constraints for sum but the implementation also supports versions for max, min, average, product and count.
  \samuel{Motivate why product, not sum (=diff) you only need one, hardcoded preference for display - diff not present - pruning relation?} 
  }

  % \sergey{Here a detailed explanation on what is going on, what is essential, what is not? etc should be self-explanatory}\samuel{Motivate why product, not sum (=diff) you only need one, hardcoded preference for display} \sergey{1) should be readable alone without lookin in the text of the paper 2) that's the constraints we found popular according the spreadsheets in the web 3) that's what is actually implemented in the system 4) corresponds to the actual Excel functions}}

  \label{table:constraints}
\end{table*}

Table~\ref{table:constraints} shows all constraint templates currently supported by our implementation.
Their dependencies are reflected in their \CSignature and shown in Figure~\ref{fig:learning_order}.
These constraints have been selected based on the most-used Excel functions as well as their occurrence in various example spreadsheets we observed.

Generally \CSignature and / or the \CFunction can be relaxed to capture more constraints in the data, however, this also comes at a performance penalty. \samuel{More on this / example?}


% \subsection{Workflow}
% \begin{algorithm}[thb]
%   \begin{algorithmic}
%     \footnotesize
%     \State \textbf{Input:} $D$ -- dataset, \constraints -- constraints, \dependencies -- dependencies \\(optional: tables $T$, groups $G$)
%     \State \textbf{Output:} $S$ -- learned constraints with their satisfaction assignment
%     \If{$T$ is \textbf{not} provided}
%       \State $T \gets \extracttables(D)$
%     \EndIf
%     \If{$G$ is \textbf{not} provided}
%       \State $G \gets \extractgroups(D, T)$
%     \EndIf
%     \State $S \gets \learnconstraints(G,\constraints,\dependencies)$
%     \State \Return $S$
% \end{algorithmic}
% \caption{Workflow}
% \label{algo:workflow}
% \end{algorithm}

% \textbf{Approach}
% \begin{itemize}
%   \item Notation
%   \item Algorithm (select constraints, find assignments, find solutions)
% \end{itemize}

% \samuel{Move next to subgroup satisfaction search}
% \section{Declarative modeling}
% \sergey{We need to fit ASP, Minizinc and all that here, I mean it is supposed to be an important point after all}

\newcommand{\runtotal}{16.12}
\newcommand{\runtotalstd}{0.62}

\newcommand{\runfile}{0.50}
\newcommand{\runfilestd}{0.02}

\section{Evaluation}
\sergey{we need to add a summary of the dataset, avg number of constraints, cells, rows, columns}
In this section we experimentally validate our approach.
We studied various questions, most notably with what accuracy our algorithm can find essential constraints.

The implementation is illustrated using a case study on the spreadsheet corresponding with the previously introduced example (figure~\ref{fig:main_example}).
In order to quantify the results and generalize our findings we also evaluate our algorithm on a benchmark of 30 (\samuel{check number}) spreadsheets that we assembled from various sources.

In this section we focus on \textit{functional} constraints that could be used in spreadsheets, ignoring constraints such as all-different or foreign-key.

All experiments were run on a Macbook Pro, Intel Core i7 2.3 GHz with 16GB RAM.

\subsection{Case study}
\tias{Move explanation of origin to first time it is introduced}
Let us illustrate our implementation using the example presented in Figure~\ref{fig:main_example}.
This example combines several smaller examples that were used in an exercise session to teach Excel into one spreadsheet.
Figure~\ref{fig:sol_example} shows the constraints that we expect to find.

\begin{figure}
  {\small
    \begin{align*}
      & SERIES(\range{T_{1}}{\rangeall}{1}) \\
%
      & \ecrank{\range{T_{1}}{\rangeall}{8}}{\range{T_{1}}{\rangeall}{7}} \\
%
      & \ecsumc{\range{T_{2}}{1}{\rangeall}}{\range{T_{1}}{\rangeall}{\rangeto{3}{7}}} \\
%
      & \ecsumc{\range{T_{6}}{\rangeall}{2}}{\range{T_{1}}{\rangeall}{\rangeto{3}{6}}} \\
%
      & \ecsumr{\range{T_{1}}{\rangeall}{7}}{\range{T_{1}}{\rangeall}{\rangeto{3}{6}}} \\
%
      & \ecaggc{AVERAGE}{\range{T_{2}}{2}{\rangeall}}{\range{T_{1}}{\rangeall}{\rangeto{3}{7}}} \\
%
      & \ecaggc{MAX}{\range{T_{2}}{3}{\rangeall}}{\range{T_{1}}{\rangeall}{\rangeto{3}{7}}} \\
%
      & \ecaggc{MIN}{\range{T_{2}}{4}{\rangeall}}{\range{T_{1}}{\rangeall}{\rangeto{3}{7}}} \\
%
      & \ecaggif{SUM}{\range{T_{1}}{\rangeall}{10}}{\range{T_{5}}{\rangeall}{1}}{\range{T_{1}}{\rangeall}{2}}{\range{T_{5}}{\rangeall}{2}} \\
%
      & \ecaggif{MAX}{\range{T_{1}}{\rangeall}{11}}{\range{T_{5}}{\rangeall}{1}}{\range{T_{1}}{\rangeall}{2}}{\range{T_{5}}{\rangeall}{2}} \\
%
      & \eclookup{\range{T_{4}}{\rangeall}{3}}{\range{T_{4}}{\rangeall}{2}}{\range{T_{1}}{\rangeall}{1}}{\range{T_{1}}{\rangeall}{2}} \\
%
      & \range{T_{6}}{\rangeall}{4} = PREV(\range{T_{6}}{\rangeall}{4}) + \range{T_{6}}{\rangeall}{2} - \range{T_{6}}{\rangeall}{3}
    \end{align*}
  }
  \caption{Constraints that are satisfied in the case study (Figure~\ref{fig:main_example}) and expected to be found by our implementation. Functional constraint are omitted.}
  \label{fig:sol_example}
\end{figure}

\subsubsection{Results}
Our current implementation takes a few seconds to find 18 constraints, including all of the 12 solutions described in Figure~\ref{fig:sol_example}.
Of the 6 remaining (redundant) constraints, 5 are $\mathit{rank}$ constraints that are true by accident, such as: \begin{align*}
  & \ecrank{\range{T_1}{\rangeall}{1}}{\range{T_1}{\rangeall}{5}}
\end{align*}
The other redundant constraint is an additional $\mathit{lookup}$ that holds because the two vectors in \range{T_2}{\rangeall}{\rangeto{2}{3}} can both be used to look each other up in \range{T_1}{\rangeall}{\rangeto{1}{2}} and we considered only one of them to be essential (lookup using the ID).

For this example our primary goal of finding all constraints is achieved.
The implementation also returns a number of redundant constraints ($33.33\%$ of the total).
This ratio is, as we will show, rather high compared to other spreadsheets.
However, this example contains many short vectors which increases the chance for constraints to be true by accident.

\subsection{Benchmark}
There are three main categories of spreadsheets in the benchmark: spreadsheets from an exercise session teaching Excel at an affiliated University, spreadsheets from tutorials online and publicly available spreadsheets such as crime statistics or financial reports that demonstrate more real world usage\footnote{\samuel{link to links.txt}}.
The case study is also included.

All spreadsheets have been converted to CSV with no manual intervention unless noted.
Definitions of tables were added manually, providing a way to compare results and overcoming shortcomings in the table detection algorithm (in some harder cases groups where also provided manually).

For every spreadsheet a ground-truth has been provided manually, specifying the essential (or original) constraints that are expected to be discovered.
We consider three types of constraints:
\begin{description}
  \item[Implemented] Constraint that have been implemented and are expected to be found (all constraints in Table~\ref{table:constraints})
  \item[Essential] Constraint that have or have not been implemented but could be found using our algorithm (e.g. fuzzy conditional sum)
  \item[Non-trivial] Constraints currently outside of the scope of this system (e.g. generic nested mathematical or logical formulas or n-ary constraints)
\end{description}

We examine three main questions concerning the accuracy, redundancy and efficiency of our approach, our main focus being accuracy.

\subsubsection*{Q1. How accurate is the approach?}
Fortunately, our system is currently able to find $100\%$ of the implemented constraints (90) on the benchmark suite.
There are only three essential constraints that have not been implemented yet, therefore our system currently identifies $96.77\%$ of the essential constraints.

\subsubsection*{Q2. How many redundant constraints are discovered?}
Our primary focus was, as mentioned before, accuracy, therefore we sometimes traded off less redundancy for more accuracy.
The motivation is that solutions can still be pruned using entailment or heuristics afterwards.
The constraints we considered to be redundant are either duplications (results that can be calculated in different ways, one of which seems superior) or constraints that hold by accident.

Across all spreadsheets our implementation finds 121 constraints, 28 ($23.14\%$) of which are redundant.
However, the average redundancy per spreadsheet is only $8.33\%$.
Examining the redundant constraints reveals that many (12) of them are duplications occurring in a single spreadsheet.
Many of the remaining constraints are duplications stemming from the overlapping role of difference and sum.
Accidental constraints are limited to the 5 rank constraints that were discussed in the case study.

Redundant constraints can be reduced in different ways.
Duplications should be detected in a post-processing step that uses entailment or heursitics to filter constraints.
Accidental constraints may be detected through heuristics, however, they can be also be removed by adding additional data.

\subsubsection*{Q3. How fast is the algorithm?}
Concerning the speed of the algorithm we also prioritized accuracy when a trade-off had to be made.
For the 32 spreadsheets in the benchmark our implementation ran in ${\runtotal}s \pm {\runtotalstd}s$.
The execution times vary widely though between spreadsheets, only four spreadsheets taking more than $0.2s$.
Most of the execution time in these cases goes towards searching either aggregate constraints or conditional aggregates.
The search for aggregates will be slow on spreadsheets containing larger groups of numeric data.
For conditional aggregates the number of candidate primary keys (all-different) and numeric vectors determines the running time (e.g. the case study example).

\begin{table}
  \centering
  \begin{tabular}{lll}
    & \textbf{Total} & \textbf{Average per spreadsheet} \\
    \textbf{Constraints} & $93$ & $2.91$ \\
    \textbf{Accuracy} & $96.77\%$ & $94.27\%$ \\
    \textbf{Redundant} & $23.14\%~(28)$ & $8.33\%~(0.88)$ \\
    \textbf{Speed (s)} & ${\runtotal}s \pm {\runtotalstd}s$ & ${\runfile}s \pm {\runfilestd}s$
  \end{tabular}
  \caption{Overview evaluation on essential constraints}
\end{table}
\samuel{Expand summary}

\paragraph{Dependencies}
In order for the algorithm to run efficiently it is crucial to use dependencies and find constraint incrementally whenever possible.
This avoids some of the explosion of combinations for constraints that have many arguments.
For example, using foreign keys as a base constraint to find conditional aggregates reduces the running time for the case study from about 3 seconds to below 1 second.
Unfortunately, this assumption is sometimes too strong, when users are interested in aggregates for only some of the keys that are present in the data.

\section{Applications}
In this section we will illustrate how various motivating applications could be realized using our system.

\subsection{Autocompletion}
Autocompletion can be seen as using knowledge derived from data at time~$t_0$ and new input data at time~$t_1 > t_0$ to predict data that the user has not yet written.
Therefore, constraints are learned at time~$t_0$ and combined with data from a new row~$i$ (column~$j$) that the user has added to a table~$T$ by time~$t_1$.
Constraints that do not conflict with the added data are considered viable.
Viable constraints that have enough input data to compute the result for a blank cell~$T[i,j]$ in the new row (column) are \textit{active}, which means they can be used to predict the outcome for that cell.
Active constraints can be used to autocomplete blank cells such that the user does not have to fill in the values manually.
Once the user has stopped editing the new row~$i$ (column~$j$), the table~$T$ is updated to include~$i$ ($j$) and constraints are learned once again.

\subsection{}
\sergey{We should explain Figure \ref{fig:fbi}, e.g. if we aggregate by the crime now, the missing value would not get into statistics, which seems faulty, since we can directly compute it.}

\begin{figure*}[thb]
  \begin{center}
    \includegraphics[width=0.85\textwidth]{figures/fbi_figure_highlighted.png}
  \end{center}
  \caption{Real world tabular constraint reconstruction: FBI crime statistics}
  \label{fig:fbi}
\end{figure*}

\samuel{FBI: work out error detection algorithm}

(\sergey{I think there should be three things: key example evaluated, benchmark of 30 spreadsheets and maybe the FBI example, how we can detect interesting dependencies and correct potential mistakes})

\samuel{discussion: what would be a language to extend the system, what would be the user interface? can it be used by community?  can it be complitely automated? interaction with user and highlighing?
}

{\bfseries
  Experimental questions
}

\begin{itemize}
  \item  How accurate are we? (Accuracy / recall)
  \item  How fast are we and which factors affect the runtime (how)?
  \item  How general is our approach, what limitations are there?
\end{itemize}


\section{Related Work}
\sergey{key bullet points for Luc and possibly Samuel and me to make related work section}

\samuel{Two possible takes here: 1) recent interest in topics such as flashfill, etc -> sparked the idea for this project. This actually relates to constraint learning which has seen an uptick recently and is a subfield of long-established ILP. 2) ILP established field trying to learn structure / .., has been brought to the less traditional context (constraint learning), learning constraints / programs has been picked up by specific approaches such as flashfill etc which inspired this research}

\sergey{ECAI reference style file ignores their guideline and their guideline ignores what is written in the guidelines!}
flashfill, flashextract, flashmeta \cite{flashfill,flashextract,flashmeta}
\begin{itemize}
  \item their supervised vs our unsupervised approach
  \item they look for a single ``smallest'' solution, we enumerate them all
  \item they are looking for a function, we solve constraint satisfaction problems
  \item we do not assume classic row based data layout, we work in the tabular setting
\end{itemize}

sketch \cite{sketch}
\begin{itemize}
  \item look for a constant that would fill in the gap in a program
  \item tailored for programming languages
  \item similar to model checking
  \item looks for a single solution
  \item similar to constraint satisfaction and sat, where one is interested in a single assignment that works for any potential input
\end{itemize}

tabular \cite{tabular}
\begin{itemize}
  \item language based on the excel tables that specify probabilistic models
  \item a system for probabilistic inference and similarity mostly in the usage of excel
  \item probabilistic constraint satisfaction (?) and graphical models
  \item single solution again
\end{itemize}

modelseeker \cite{modelseeker} \sergey{Samuel, Luc, probably you would need elaborate here more in details}

\begin{itemize}
  \item not designed for excel-like data representation (type consistency, groups, etc)
  \item not designed for excel-like constraints (lookups, conditional ifs, etc)
  \item does not support user extensions (?)
\end{itemize}

claudien \cite{claudien} \sergey{Samuel, Luc, you would need to help with this one}

\section{Conclusions}
\sergey{link to the github with implementations and the dataset}


\sergey{Future version, extensions, declarative stuff, bla-bla here}

% points: 1) can be a base for an actual plugin or a function in excel 2) novel problem and challenge for systems and constraint solvers 3) can be part of the complicated pipeline togehter with the header detection
Our goal was to automatically identify constraints that hold in a spreadsheet.
We presented and evaluated an approach that is able to learn a large set of constraints from CSV data.
Moreover, the amount of redundant constraints found by the algorithm is limited despite the lack of more sophisticated post-processing steps.

\paragraph{Future work}
We see multiple directions for future work on this topic.
The system we presented would become more useful in an interactive setting where users can easily receive and provide feedback.
Moreover, the system could be adapted to produce less redundant constraints through the use of heuristic filtering.

Currently, the system only learns single constraints, however, extending this approach to nested constraints would allow more expressive concepts to be learned.
This shift would bring the approach more in line with programming by example.

Aside from finding errors, the system could be extended to deal with noise and the ability to also learn soft constraints.
Soft constraints are (potentially weighted) constraints that hold only on some of the data.
This could extend the approach to new application domains as well as provide more native error detection.

\bibliographystyle{ecai}
\bibliography{references}
\end{document}
%%%%%%%%%%%%%%%%%%%%%%%%%%%%%%%%%%%%%%%%%%%%%%%%%%%%%%%%%%%%%%%%%%%%%%
